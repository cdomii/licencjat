\documentclass[12pt, a4paper, titlepage]{report}
\usepackage{polski}
\usepackage[utf8]{inputenc}
\usepackage[T1]{fontenc}
\usepackage{indentfirst}
\setlength{\parindent}{1cm}
\linespread{1.3}


%opening
\title{\textbf{Wizerunek subkultury heavymetalowej}}
\author{Dominika Cygankiewicz}
\date{}

\begin{document}
	
\maketitle
\tableofcontents
\thispagestyle {empty}
%\setlength{\parskip}{1ex plus 0.5ex minus 0.2ex}
\newpage

\chapter*{Wstęp}
\addcontentsline{toc}{chapter}{Wstęp}

\chapter{Subkultura heavymetalowa}
\section{Czym jest subkultura?}
Termin "subkultura" po raz pierwszy pojawił się w latach 80. XIX wieku w dziedzinie biologii, gdzie służył do opisywania kultur mikroorganizmów.\footnote{Krótka historia młodzieżowej subkulturowości} W obszarze nauk społecznych zaczął funkcjonować dopiero w latach 40. XX wieku, za sprawą Miltona M. Gordona, który w artykule \textit{The Concept of the Sub-Cultur and Its Application} pisał o subkulturze jako o "pododdziale kultury narodowej". \footnote{The Concept of the Sub-Cultur and Its Application} M. M. Gordon, konieczność wprowadzenia pojęcia "subkultura" upatrywał w braku istniejącego terminu charakteryzującego pomniejsze kultury, wykształcające się w określonych grupach społecznych: ekonomicznych, regionalnych, etnicznych i religijnych. W jego rozumieniu, tak uformowane subkultury, miały znaczący wpływ na, przynależące do nich i uczestniczące w nich, poszczególne jednostki. %sprawdzić

% podejmowano różnych prób zdefiniowania zjawiska. Napotkane przez badaczy, trudności definicyjne mogły świadczyć o złożoności i wielowątkowości prezentowanego problemu.
%Mówiąc o rozwoju zainteresowania socjologów przedmiotem subkulturowości, Witold Wrzesień wyróżnia 3 okresy studiów subkulturowych.

Zaproponowana definicja, okazała się być jednak jedynie pierwszym przystankiem na drodze do wyjaśnienia zjawiska. W kolejnych latach studiów subkulturowych, podejmowano się wielu prób analizy powstających subkultur i dążono do wypracowania możliwie kompleksowej definicji. %Praca M. M. Gordona, spotkała się z krytyką.

Przez długi czas, subkultury były kojarzone z odstępstwami od przyjętych norm kulturowych.\footnote{Mały słownik subkultur młodzieżowych} Już etymologia słowa wskazuje pewne wzorce interpretacyjne pojęcia. Przedrostek "sub-" niejako sugeruje istnienie kulturowej hierarchii, w której to subkultura plasuje się pod kulturą \textit{nadrzędną}. Jednocześnie zaznacza pochodność subkultury od kultury masowej. %pejoratywne znaczenie terminu. 


\newpage
\section{Geneza subkultury heavymetalowej}
%zastanowić się nad przeniesieniem tego podrozdziału na sam koniec

\section{Historia heavy metalu na świecie}
Powstanie terminu "heavy metal" datuje się nawet na dwieście lat wstecz. 
\section{Historia heavy metalu w Polsce}
\chapter{Wizerunek subkultury heavymetalowej w mediach}
\section{Definicja wizerunku}
\section{Muzyka heavymetalowa i jej fani w mediach zagranicznych}
\section{Muzyka heavymetalowa i jej fani w mediach polskich}
\section{Heavy metal w social media}
\chapter{Badanie wizerunku subkultury heavymetalowej}
\section{Przedmiot badań i cele badawcze}
\section{Metody i narzędzia badawcze}
\section{Analiza wyników badania}
\chapter*{Zakończenie}
\addcontentsline{toc}{chapter}{Zakończenie}
\chapter*{Bibliografia}
\addcontentsline{toc}{chapter}{Bibliografia}

\end{document}
