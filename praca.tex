\documentclass[12pt, a4paper, titlepage]{report}
\usepackage{polski}
\usepackage[utf8]{inputenc}
\usepackage[T1]{fontenc}
\usepackage{indentfirst}
\setlength{\parindent}{0.8cm}
\linespread{1.3}
\usepackage{times}

%opening
\title{\textbf{Wizerunek subkultury heavymetalowej \break w Polsce}}
\author{Dominika Cygankiewicz}
\date{}

\begin{document}
	
\maketitle
\tableofcontents
\thispagestyle {empty}
%\setlength{\parskip}{1ex plus 0.5ex minus 0.2ex}
\newpage

\chapter*{Wstęp}
\addcontentsline{toc}{chapter}{Wstęp}

\chapter{Subkultura heavymetalowa}
\section{Czym jest subkultura?}
Termin "subkultura" po raz pierwszy pojawił się w latach 80. XIX wieku w dziedzinie biologii, gdzie służył do opisywania kultur mikroorganizmów.\footnote{Witold Wrzesień, Krótka historia młodzieżowej subkulturowości, Warszawa 2013, PWN, \break str. 40} W obszarze nauk społecznych zaczął funkcjonować dopiero w latach 40. XX wieku, za sprawą Miltona M. Gordona, który w artykule \textit{The Concept of the Sub-Cultur and Its Application} pisał o subkulturze jako "pododdziale kultury narodowej". \footnote{Milton Gordon, The Concept of the Sub-Cultur and Its Application} M. M. Gordon, konieczność wprowadzenia pojęcia "subkultura" upatrywał w braku istniejącego terminu charakteryzującego pomniejsze kultury, wykształcające się w określonych grupach społecznych: ekonomicznych, regionalnych, etnicznych i religijnych. W jego rozumieniu, tak uformowane subkultury, miały znaczący wpływ na, przynależące do nich i uczestniczące w nich, poszczególne jednostki.\footnotemark[\value{footnote}] %sprawdzić

% podejmowano różnych prób zdefiniowania zjawiska. Napotkane przez badaczy, trudności definicyjne mogły świadczyć o złożoności i wielowątkowości prezentowanego problemu.
%Mówiąc o rozwoju zainteresowania socjologów przedmiotem subkulturowości, Witold Wrzesień wyróżnia 3 okresy studiów subkulturowych.

Był to pierwszy przystanek na drodze do wyjaśnienia zjawiska. Sformułowana przez M. M. Gordona teza, spotkała się z krytyką i przez kolejnych badaczy, uznana została za zbyt szeroką i niejasną.\footnote{Witold Wrzesień, Krótka historia młodzieżowej subkulturowości, Warszawa 2013, PWN, \break str. ?} W kolejnych latach studiów subkulturowych, dążono do wypracowania możliwie jak najbardziej kompleksowej definicji. Wielość prac badawczych oznaczała jednak wiele różnorodnych punktów widzenia i prób ujęcia badanego problemu. %Praca M. M. Gordona, spotkała się z krytyką.

Przez długi czas, subkultury były kojarzone z odstępstwami od przyjętych norm kulturowych i buntem.\footnote{Mirosław Pęczak, Mały słownik subkultur młodzieżowych, Warszawa 1992, Semper, str. 3} W rozumieniu potocznym, wciąż przywodzić mogą na myśl patologie i brak przystosowania społecznego.\footnote{Marian Filipiak, Od subkultury do kultury alternatywnej} Przyczyn przestępczości wśród nieletnich często upatruje się między innymi w przynależności młodych ludzi do subkultury. Dla służb porządkowych wprowadzane są specjalne podręczniki traktujące o polskich subkulturach. Przykładem takiej publikacji jest praca Marcela Śmiałka \textit{Subkultury młodzieżowe w Polsce. Wybrane zagadnienia} z 2015 roku. Podręcznik w skrócie charakteryzuje subkultury występujące w Polsce i opisuje przyczyny ich powstawania. Wymienia się wśród nich między innymi brak perspektyw samorealizacji, zaburzenia rodzinnych więzi emocjonalnych oraz destrukcyjne formy zachowań propagowane przez media.\footnote{Marcel Śmiałek, Subkultury młodzieżowe w Polsce. Wybrane zagadnienia, Legionowo 2015, Wydział Wydawnictw i Poligrafii, str. 12} Taka kategoryzacja, stawia członków subkultur w mało przychylnym świetle. 

Z utożsamianiem subkultur młodzieżowych z nieprzystosowaniem społecznym nie zgadza się (?)

%na koniec to raczej
Etymologia słowa również wskazuje pewne wzorce interpretacyjne pojęcia. Przedrostek "sub-" sugeruje pochodność subkultury od kultury.   Należy tu nadmienić, że wyraz "subkultura" wywodzący się z angielskiego słowa \textit{subculture}, w tłumaczeniu na język polski, oznacza tyle, co \textit{podkultura}. Termin ten, w języku polskim, budzi konotacje nacechowane negatywnie.\footnote{Mirosław Pęczak} Silniej zaznacza istnienie kulturowej hierarchii, w której to podkultura plasuje się pod kulturą \textit{nadrzędną}. Określenie "subkultura" posiada mniej deprecjonujący charakter i jest emocjonalnie neutralne. (...) Część badaczy, rozróżnia jednak subkulturę od podkultury, uznawszy je za dwa różne pojęcia. %pejoratywne znaczenie terminu. 


\newpage
\section{Geneza subkultury heavymetalowej}
%zastanowić się nad przeniesieniem tego podrozdziału na sam koniec
"Metalowcy" tworzą subkulturę fanów muzyki heavymetalowej, będącej odmianą rocka.\footnote{Mirosław Pęczak, Mały słownik subkultur młodzieżowych, Warszawa 1992, Semper, str. 53}  Największe znaczenie i popularność, zyskali w latach 80-ych, choć początki istnienia subkultury można było zaobserwować już dziesięć lat wcześniej.\footnote{Marian Filipiak, Od subkultury do kultury alternatywnej, Lublin 2003, Wydawnictwo UMCS, str. 78} 

W Polsce, pojawili się wraz z pierwszymi ugrupowaniami muzycznymi grającymi muzykę z gatunku heavy metal (na przykład KAT i TSA).\footnote{Witold Wrzesień, Krótka historia młodzieżowej subkulturowości, Warszawa 2013, PWN, str. 328} Już wtedy, spotykali się z krytycznym nastawieniem ze strony polskiego społeczeństwa. Oskarżano ich o tendencje satanistyczne i neopogańskie, oraz wyrażano głęboką dezaprobatę dla wykonywanego przez nich rodzaju muzyki.\footnote{Marian Filipiak, Od subkultury do kultury alternatywnej, Lublin 2003, Wydawnictwo UMCS, str. 78} % Zachowania polskich zespołów heavymetalowych umacniały przeciwników subkultury w tych osądach. Za przykład mogą posłużyć podpalenie i profanacja symboli religijnych na scenie, podczas festiwalu w Jarocinie w 1986 roku.\footnote{Witold Wrzesień, Krótka historia młodzieżowej subkulturowości, Warszawa 2013, PWN, str. 328}
Podstawę do nietolerancji od zawsze tworzył także wygląd fanów muzyki metalowej. Nie tylko ciężkie brzmienia i zachowania sceniczne, ale również ubiór i fryzury, formowały społeczną przestrzeń wychodzącą poza ramy koncertowych hal, i prowokowały (oraz być może, prowokują do dziś) negatywne reakcje tych, którzy do niej nie należeli. 
 
Styl heavymetalowców to przede wszystkim suma adaptacji wybranych elementów wyglądu zewnętrznego punków i motocyklistów.\footnote{Witold Wrzesień, Krótka historia młodzieżowej subkulturowości, Warszawa 2013, PWN, \break str. 323-324} Charakterystyczną \break część garderoby stanowią dla nich dżinsowe katany oraz skórzane kurtki motocyklowe (zwane potocznie ramoneskami od nazwy zespołu punkrockowego Ramones). W połączeniu z ciasnymi, dżinsowymi lub skórzanymi spodniami oraz długimi włosami, tworzą podstawę wyglądu heavymetalowców. Równie elementarne dla fanów metalu są pieszczochy - skórzane bransolety, zwykle nabijane rzędami stalowych ćwieków. W zależności od ich ilości, pieszczocha pokrywa jedynie nadgarstek lub nawet całe przedramię właściciela. Popularną ozdobą ubioru są naszywki z logotypami zespołów metalowych, a także żelazne krzyże. T-shirty, najczęściej czarne, mogą zawierać emblematy z okładek ich płyt\footnotemark[\value{footnote}] oraz rozpiskę trasy koncertowej. %pociągnąc o koncertach tutaj?
Z czasem, wśród fanów metalu powszechnym obuwiem stały się glany (ciężkie buty robocze z metalowymi noskami). 

Wizerunek ten, wspólny jest zarówno dla mężczyzn, jak i kobiet przynależących do subkultury heavymetalowej. Strój kobiecy niemal niczym nie odbiega od ubioru mężczyzn, a o samej subkulturze mówi się, jakoby zakładała kult siły i męskości.\footnotemark[\value{footnote}] Omawiany gatunek muzyki gromadzi głównie męskich odbiorców. Żeńskie zespoły metalowe należą do mniejszości. ???

Dla subkultury heavymetalowej najważniejszy jest koncert. Deena Weinstein podkreśla, że to podczas wydarzeń muzycznych, objawia się ona w idealnej postaci, a koncert heavymetalowy badaczka przyrównuje do celebracji na miarę ceremonii religijnych.\footnote{Deena Weinstein, str. 223-231} Poza tą sferą, metalowcy nie posiadają własnych rytuałów; to koncerty tworzą dla nich przestrzeń, w której najlepiej mogą manifestować swą subkulturowość.\footnote{Witold Wrzesień, str. 325} Jak sugeruje Barbara Major, wspólnotowość heavy metalu ugruntowana jest w muzyce.\footnote{Barbara Major, Dionizos w glanach. Ekstatyczność muzyki metalowej, Kraków 2013, Księgarnia Akademicka, str. 150} Koncert tworzy okazję do poznania nowych ludzi oraz, być może przede wszystkim, wymiany doświadczeń i nazw nowych, heavymetalowych kapel.\footnote{Magdalena Waga, Jak cię widzą, tak cię piszą... - czyli o metalowcach, www.epiotrkow.pl\break /artykul/Jak-cie-widza-tak-cie-pisza...--czyli-o-metalowcach,4963} Spajającym subkulturę czynnikiem jest ten sam gust muzyczny, a nie prezentowane poglądy - metalowcy nie wyznają bowiem specyficznej tylko dla nich ideologii.\footnote{Marian Filipiak, Od subkultury do kultury alternatywnej, Lublin 2003, Wydawnictwo UMCS, str. 78}

Charakterystyczne dla koncertów heavymetalowych są podejmowane przez ich uczestników formy zabawy.\footnote{Witold W, str. 324} Jedną z najpopularniejszych i jednocześnie najbardziej sztampową jest tzw. headbanging, czyli energiczne machanie, kręcenie głową w rytm muzyki. Od tego rodzaju tańca, pochodzą też anglojęzyczne nazwy subkultury - \textit{headbangers} i \textit{metalheads}.\footnote{Witold Wrzesień, str. 325} Innym rodzajem tanecznej aktywności jest pogo, typowe również dla subkultury punkowej. Jego uczestnicy wzajemnie odbijają się od siebie, skacząc i energicznie potrząsając przy tym głową.\footnote{Barbara Major, Dionizos w glanach..., Kraków 2013, Księgarnia Akademicka, str. 158} Przestrzeń, w której odbywa się ten taniec, zwykle zwie się \textit{kotłem} bądź \textit{młynem}; i znajduje się ona w środkowej i jednocześnie najbardziej żywo reagującej części widowni. Reprezentatywną dla metalowców odmianą pogo jest tzw. \textit{mosh}, w którym bardzo dużą rolę odgrywają ruchy głowy. W \textit{moshingu} dozwolone jest używanie pieszczoch oraz glanów. Uważa się, że ten rodzaj pogo jest bardziej agresywny niż pogo punkowe, ale jednocześnie wydaje się być mniej dynamiczne.\footnote{Anna Matras, Pomiędzy ciałem indywidualnym a zbiorowym. Praktyki cielesne punków i ich strategie uczestnictwa w Festiwalu w Jarocinie, Palimpsest nr 2, marzec 2012, str. 154} Popularną formą zabawy w pogo jest dla subkulturowości metalowej również \textit{ściana śmierci}. Jej uczestnicy stają na przeciw siebie w dwóch rzędach i w odpowiednim momencie wykonywanego przez grający zespół utworu, ruszają wprost na siebie.\footnote{Barbara Major, Dionizos w glanach..., Kraków 2013, Księgarnia Akademicka, str. 158} 

Przybliżone wyżej formy koncertowej zabawy mogą sprawiać wrażenie agresywnych, a nawet niebezpiecznych. 

Dla społeczności heavymetalowej, metal jest nie tylko gatunkiem muzycznym. Bycie fanem metalu wiąże się z zaangażowaniem oraz przyjęciem określonego sposobu życia; nadaje mu znaczenie i kreuje poczucie przynależności.\footnote{William Phillips, Brian Cogan, ..., str. 5-6}

\section{Historia heavy metalu na świecie}
Powstanie terminu "heavy metal" datuje się nawet na dwieście lat wstecz. W XIX wieku, pod pojęciem heavy metalu rozumiano ciężką artylerię wojskową, a wyrażenie \textit{a man of heavy metal} oznaczało człowieka silnego, posiadającego władzę.\footnote{Robert Walser, Running with the Devil: Power, Gender, and Madness in Heavy Metal Music, 2014, Wesleyan University Press}  Obecnie, według \textit{Cambridge Dictionary} wyróżnia się jego dwa znaczenia: 
\begin{enumerate}
\item w chemii i przemyśle metalurgicznym: ciężkie, często toksyczne metale  i półmetale
\item w muzyce: rodzaj muzyki rockowej o mocnym brzmieniu
\end{enumerate}
%gdzieś tutaj footnote
Pochodzenie określenia "heavy metal" w kontekście muzyki rockowej, pozostaje kwestią sporną. Istnieje kilka skrajnie różnych stanowisk w tej kwestii. Część badaczy przyjmuje, że pojawiło się za sprawą powieści Williama S. Burroughsa, w której autor jednego z bohaterów określił mianem "the heavy metal kid". Za inne potencjalne źródło terminu podaje się utwór "Born to Be Wild" zespołu Steppenwolf.\footnote{Barbara Major, Dionizos w glanach..., str. 41} Niezależnie od tego, jaki punkt widzenia przyjmiemy, termin został rozpowszechniony we wczesnych latach 70. i szybko stał się nazwą dla nowego, cięższego gatunku muzyki rockowej. 
 
Do pojawienia się heavy metalu przyczyniły się dwa zespoły: Black Sabbath oraz Led Zeppelin.\footnote{William Phillips, Brian Cogan, Encyclopedia of Heavy Metal Music, Westport 2009, ABC-CLIO, str. 28} Wpłynęły nie tylko na rozwój gatunku muzycznego, ale także na wykształcony w późniejszych latach styl ubioru i sposób postrzegania świata przez fanów i wykonawców metalu. 

\section{Historia heavy metalu w Polsce}
Heavy metal w Polsce pojawił się na początku lat 80-ych, wraz ze wzrostem popularności muzyki rockowej w kraju. %pęczak, ale czy pisać?
\chapter{Wizerunek subkultury heavymetalowej w mediach}
\section{Definicja wizerunku}
\section{Muzyka heavymetalowa i jej fani w mediach zagranicznych}
\section{Muzyka heavymetalowa i jej fani w mediach polskich}
\section{Heavy metal w social media}
\chapter{Badanie wizerunku subkultury heavymetalowej}
\section{Przedmiot badań i cele badawcze}
\section{Metody i narzędzia badawcze}
\section{Analiza wyników badania}
\chapter*{Zakończenie}
\addcontentsline{toc}{chapter}{Zakończenie}
\chapter*{Bibliografia}
\addcontentsline{toc}{chapter}{Bibliografia}

\end{document}
