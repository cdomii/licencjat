\documentclass[12pt, a4paper, titlepage]{report}
\usepackage{polski}
\usepackage[utf8]{inputenc}
\usepackage[T1]{fontenc}
\usepackage{indentfirst}
\setlength{\parindent}{1cm}
\linespread{1.3}


%opening
\title{\textbf{Wizerunek subkultury heavymetalowej}}
\author{Dominika Cygankiewicz}
\date{}

\begin{document}
	
\maketitle
\tableofcontents
\thispagestyle {empty}
%\setlength{\parskip}{1ex plus 0.5ex minus 0.2ex}
\newpage

\chapter*{Wstęp}
\addcontentsline{toc}{chapter}{Wstęp}

\chapter{Subkultura heavymetalowa}
\section{Czym jest subkultura?}
Termin "subkultura" po raz pierwszy pojawił się w latach 80. XIX wieku w dziedzinie biologii, gdzie służył do opisywania kultur mikroorganizmów.\footnote{Krótka historia młodzieżowej subkulturowości} W obszarze nauk społecznych zaczął funkcjonować dopiero w latach 40. XX wieku, i od tego czasu, wciąż nie udało się stworzyć jednej, spójnej definicji zjawiska, choć wielu badaczy podejmowało się prób jej sformułowania. %ale napotykało trudności definicyjne? nie potrafili wcisnąć tak złożonego problemu w kilka zdań? myśl :C
Świadczyć to może nie tylko o złożoności  problemu, ale także o jego powszechności. %popularności? dziwne zdanie
%Mówiąc o rozwoju zainteresowania socjologów przedmiotem subkulturowości, Witold Wrzesień wyróżnia 3 okresy studiów subkulturowych.




\newpage
\section{Geneza subkultury heavymetalowej}
%zastanowić się nad przeniesieniem tego podrozdziału na sam koniec

\section{Historia heavy metalu na świecie}
Powstanie terminu "heavy metal" datuje się nawet na dwieście lat wstecz. 
\section{Historia heavy metalu w Polsce}
\chapter{Wizerunek subkultury heavymetalowej w mediach}
\section{Definicja wizerunku}
\section{Muzyka heavymetalowa i jej fani w mediach zagranicznych}
\section{Muzyka heavymetalowa i jej fani w mediach polskich}
\section{Heavy metal w social media}
\chapter{Badanie wizerunku subkultury heavymetalowej}
\section{Przedmiot badań i cele badawcze}
\section{Metody i narzędzia badawcze}
\section{Analiza wyników badania}
\chapter*{Zakończenie}
\addcontentsline{toc}{chapter}{Zakończenie}
\chapter*{Bibliografia}
\addcontentsline{toc}{chapter}{Bibliografia}

\end{document}
