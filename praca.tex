\documentclass[12pt, a4paper, titlepage]{report}
\usepackage{polski}
\usepackage[utf8]{inputenc}
\usepackage[T1]{fontenc}
\usepackage{indentfirst}
\setlength{\parindent}{1cm}
\linespread{1.3}
\usepackage{times}

%opening
\title{\textbf{Wizerunek subkultury heavymetalowej}}
\author{Dominika Cygankiewicz}
\date{}

\begin{document}
	
\maketitle
\tableofcontents
\thispagestyle {empty}
%\setlength{\parskip}{1ex plus 0.5ex minus 0.2ex}
\newpage

\chapter*{Wstęp}
\addcontentsline{toc}{chapter}{Wstęp}

\chapter{Subkultura heavymetalowa}
\section{Czym jest subkultura?}
Termin "subkultura" po raz pierwszy pojawił się w latach 80. XIX wieku w dziedzinie biologii, gdzie służył do opisywania kultur mikroorganizmów.\footnote{Krótka historia młodzieżowej subkulturowości} W obszarze nauk społecznych zaczął funkcjonować dopiero w latach 40. XX wieku, za sprawą Miltona M. Gordona, który w artykule \textit{The Concept of the Sub-Cultur and Its Application} pisał o subkulturze jako "pododdziale kultury narodowej". \footnote{The Concept of the Sub-Cultur and Its Application} M. M. Gordon, konieczność wprowadzenia pojęcia "subkultura" upatrywał w braku istniejącego terminu charakteryzującego pomniejsze kultury, wykształcające się w określonych grupach społecznych: ekonomicznych, regionalnych, etnicznych i religijnych. W jego rozumieniu, tak uformowane subkultury, miały znaczący wpływ na, przynależące do nich i uczestniczące w nich, poszczególne jednostki. %sprawdzić

% podejmowano różnych prób zdefiniowania zjawiska. Napotkane przez badaczy, trudności definicyjne mogły świadczyć o złożoności i wielowątkowości prezentowanego problemu.
%Mówiąc o rozwoju zainteresowania socjologów przedmiotem subkulturowości, Witold Wrzesień wyróżnia 3 okresy studiów subkulturowych.

Był to pierwszy przystanek na drodze do wyjaśnienia zjawiska. Sformułowana przez M. M. Gordona teza, spotkała się z krytyką i przez kolejnych badaczy, uznana została za zbyt szeroką i niejasną.\footnote{Krótka historia młodzieżowej subkulturowości} W kolejnych latach studiów subkulturowych, dążono do wypracowania możliwie jak najbardziej kompleksowej definicji. Wielość prac badawczych oznaczała jednak wiele różnorodnych punktów widzenia i prób ujęcia badanego problemu. %Praca M. M. Gordona, spotkała się z krytyką.

Przez długi czas, subkultury były kojarzone z odstępstwami od przyjętych norm kulturowych i buntem.\footnote{Mirosław Pęczak, Mały słownik subkultur młodzieżowych, Warszawa 1992, Semper, str. 3} W rozumieniu potocznym, wciąż przywodzić mogą na myśl patologie i brak przystosowania społecznego.\footnote{Marian Filipiak, Od subkultury do kultury alternatywnej} O negatywnych skojarzeniach z subkulturowością, świadczą również specjalne podręczniki wprowadzone dla służb porządkowych, które traktują o polskich subkulturach. Przykładem takiej publikacji jest praca Marcela Śmiałka \textit{Subkultury młodzieżowe w Polsce. Wybrane zagadnienia} z 2015 roku. Podręcznik w skrócie charakteryzuje subkultury występujące w Polsce i opisuje przyczyny ich powstawania. Wymienia się wśród nich między innymi brak perspektyw samorealizacji, zaburzenia rodzinnych więzi emocjonalnych oraz destrukcyjne formy zachowań propagowane przez media.\footnote{Marcel Śmiałek, Subkultury młodzieżowe w Polsce. Wybrane zagadnienia, Legionowo 2015, Wydział Wydawnictw i Poligrafii, str. 12} 

Z utożsamianiem subkultur młodzieżowych z nieprzystosowaniem społecznym nie zgadza się (?)

%na koniec to raczej
Etymologia słowa również wskazuje pewne wzorce interpretacyjne pojęcia. Przedrostek "sub-" sugeruje pochodność subkultury od kultury.   Należy tu nadmienić, że wyraz "subkultura" wywodzący się z angielskiego słowa \textit{subculture}, w tłumaczeniu na język polski, oznacza tyle, co \textit{podkultura}. Termin ten, w języku polskim, budzi konotacje nacechowane negatywnie. Silniej zaznacza istnienie kulturowej hierarchii, w której to podkultura plasuje się pod kulturą \textit{nadrzędną}. Określenie "subkultura" posiada mniej deprecjonujący charakter i jest emocjonalnie neutralne. (...) Część badaczy, rozróżnia jednak subkulturę od podkultury, uznawszy je za dwa różne pojęcia. %pejoratywne znaczenie terminu. 


\newpage
\section{Geneza subkultury heavymetalowej}
%zastanowić się nad przeniesieniem tego podrozdziału na sam koniec
"Metalowcy" tworzą subkulturę fanów muzyki heavymetalowej, będącej odmianą rocka. 

\section{Historia heavy metalu na świecie}
Powstanie terminu "heavy metal" datuje się nawet na dwieście lat wstecz. W XIX wieku, pod pojęciem heavy metalu rozumiano ciężką artylerię wojskową, a wyrażenie \textit{a man of heavy metal} oznaczało człowieka silnego, posiadającego władzę.\footnote{Robert Walser, Running with the Devil: Power, Gender, and Madness in Heavy Metal Music, 2014, Wesleyan University Press}  Obecnie, według \textit{Cambridge Dictionary} wyróżnia się jego dwa znaczenia: 
\begin{enumerate}
\item w chemii i przemyśle metalurgicznym: ciężkie, często toksyczne metale 
\item w muzyce: rodzaj muzyki rockowej o mocnym brzmieniu
\end{enumerate}
%gdzieś tutaj footnote
W swojej pracy, R. Walser wskazuje na istniejące konotacje pomiędzy dawnym rozumieniem terminu a obecną muzyką heavymetalową. 
%może to gdzieś wplątać w tekst?

\section{Historia heavy metalu w Polsce}
\chapter{Wizerunek subkultury heavymetalowej w mediach}
\section{Definicja wizerunku}
\section{Muzyka heavymetalowa i jej fani w mediach \break zagranicznych}
\section{Muzyka heavymetalowa i jej fani w mediach \break polskich}
\section{Heavy metal w social media}
\chapter{Badanie wizerunku subkultury heavymetalowej}
\section{Przedmiot badań i cele badawcze}
\section{Metody i narzędzia badawcze}
\section{Analiza wyników badania}
\chapter*{Zakończenie}
\addcontentsline{toc}{chapter}{Zakończenie}
\chapter*{Bibliografia}
\addcontentsline{toc}{chapter}{Bibliografia}

\end{document}
