\documentclass[12pt, a4paper, titlepage]{report}
\usepackage{polski}
\usepackage[utf8]{inputenc}
\usepackage[T1]{fontenc}
\usepackage{indentfirst}
\setlength{\parindent}{0.8cm}
\linespread{1.3}
\usepackage{times}

%opening
\title{\textbf{Wizerunek subkultury heavymetalowej w~Polsce}}
\author{Dominika Cygankiewicz}
\date{}

\begin{document}
	
\maketitle
\tableofcontents
\thispagestyle {empty}
%\setlength{\parskip}{1ex plus 0.5ex minus 0.2ex}
\newpage

\chapter*{Wstęp}
\addcontentsline{toc}{chapter}{Wstęp}

\chapter{Subkultura heavymetalowa}
\section{Czym jest subkultura?}
Termin "subkultura" po raz pierwszy pojawił się w~latach 80. XIX wieku w~dziedzinie biologii, gdzie służył do opisywania kultur mikroorganizmów.\footnote{Witold Wrzesień, Krótka historia młodzieżowej subkulturowości, Warszawa 2013, PWN, \break str. 40} W obszarze nauk społecznych zaczął funkcjonować dopiero w~latach 40. XX wieku, za sprawą Miltona M. Gordona, który w~artykule \textit{The Concept of the Sub-Cultur and Its Application} pisał o subkulturze jako "pododdziale kultury narodowej". \footnote{Milton Gordon, The Concept of the Sub-Cultur and Its Application} M. M. Gordon, konieczność wprowadzenia pojęcia "subkultura" upatrywał w~braku istniejącego terminu charakteryzującego pomniejsze kultury, wykształcające się w~określonych grupach społecznych: ekonomicznych, regionalnych, etnicznych i~religijnych. W jego rozumieniu, tak uformowane subkultury, miały znaczący wpływ na, przynależące do nich i uczestniczące w~nich, poszczególne jednostki.\footnotemark[\value{footnote}] %sprawdzić

% podejmowano różnych prób zdefiniowania zjawiska. Napotkane przez badaczy, trudności definicyjne mogły świadczyć o złożoności i~wielowątkowości prezentowanego problemu.
%Mówiąc o rozwoju zainteresowania socjologów przedmiotem subkulturowości, Witold Wrzesień wyróżnia 3 okresy studiów subkulturowych.

Był to pierwszy przystanek na drodze do wyjaśnienia zjawiska. Sformułowana przez M. M. Gordona teza, spotkała się z krytyką i~przez kolejnych badaczy, uznana została za zbyt szeroką i~niejasną.\footnote{Witold Wrzesień, Krótka historia młodzieżowej subkulturowości, Warszawa 2013, PWN, \break str. ?} W kolejnych latach studiów subkulturowych, dążono do wypracowania możliwie jak najbardziej kompleksowej definicji. Wielość prac badawczych oznaczała jednak wiele różnorodnych punktów widzenia i~prób ujęcia badanego problemu. 

Przez długi czas, subkultury były kojarzone z odstępstwami od przyjętych norm kulturowych i~rewoltą.\footnote{Mirosław Pęczak, Mały słownik subkultur młodzieżowych, Warszawa 1992, Semper, str. 3} Pierwotnie, w socjologii mianem subkulturowości określało się grupy nieprzystosowane społecznie i spatologizowane. W roku 1955 Albert Cohen, ich istnienie argumentował poszukiwaniem wśród zbiorowości składającej się z indywidualnych jednostek, rozwiązania problemów wynikających z prób adaptacji.\footnote{Albert K. Cohen, Deliquent Boys, The Culture of the Gang, 1995, New York: The Free Press, str. 148} Choć na przestrzeni lat badań socjologicznych, obok jego teorii wykształciło się wiele kontrastowych twierdzeń, w rozumieniu potocznym, subkultury wciąż na myśl przywodzić mogą patologie i~dewiacje.\footnote{Marian Filipiak, Od subkultury do kultury alternatywnej} Obecnie, w~przynależności młodych ludzi do subkultur często upatruje się przyczyny przestępczości wśród nieletnich. Dla służb porządkowych wprowadzane są specjalne podręczniki traktujące o polskich subkulturach, w~skrócie charakteryzujące subkultury występujące w~Polsce i~przyczyny ich powstawania. Wśród nich wymienia się między innymi brak perspektyw samorealizacji, zaburzenia rodzinnych więzi emocjonalnych oraz destrukcyjne formy zachowań propagowane przez media.\footnote{Marcel Śmiałek, Subkultury młodzieżowe w~Polsce. Wybrane zagadnienia, Legionowo 2015, Wydział Wydawnictw i~Poligrafii, str. 12} Z raportów policyjnych oraz podobnych podręczników, wynikałoby zatem, że przynależność do grup subkulturowych wiąże się z niewłaściwym przebiegiem procesu socjalizacji. Taka kategoryzacja, stawia członków subkultur w~mało przychylnym świetle. 

Etymologia słowa również wskazuje pewne negatywne wzorce interpretacyjne pojęcia i przyczynia się do takiegoż rozumienia jego sensu. Słowo "subkultura" jest odpowiednikiem angielskiego terminu \textit{subculture}, który w~tłumaczeniu na język polski, oznacza tyle, co \textit{podkultura}. Określenie to, w~języku polskim budzi konotacje nacechowane pejoratywne.\footnote{Mirosław Pęczak} Silniej zaznacza istnienie kulturowej hierarchii, w~której to podkultura plasuje się pod kulturą \textit{nadrzędną}. Termin "subkultura" posiada mniej deprecjonujący charakter i~jest emocjonalnie neutralny, choć przedrostek "sub-" bezsprzecznie sugeruje pochodność subkultury od kultury. By uniknąć wartościowania, obowiązującym określeniem w języku polskim pozostaje "subkultura". Jak podaje Marian Filipak, obejmuje ono swoim zakresem, pojawiające się w rozprawach naukowych przy temacie subkulturowości, pojęcia "kontrkultury" i "kultury alternatywnej".\footnote{M. Filipak, Od subkultury do kultury alternatywnej, str. 16}
%coś o tym? 

Termin kontrkultura pojawił się pod koniec lat 60. XX wieku, wraz z początkiem zjawiska kontestacji, i oznaczał sprzeciw wobec zastanej kultury.\footnote{M. Filipak, Od subkultury do kultury alternatywnej, str. 14} Z czasem, kontrkultura zaczęła także wytwarzać w jej miejsce nowe wzorce kulturowe -- formowała kulturę alternatywną. Przez część socjologów, subkultura uznawana jest za fazę wstępną procesu, którego rezultat stanowi kultura alternatywna.\footnote{M. Pęczak, Mały słownik subkultur młodzieżowych, str. 4}


%Analiza naukowa winna być jednak pozbawiona czynników wartościujących. 
%W próbie zdefiniowania zjawiska subkulturowości nie jest to jednak wyznacznik wystarczający. 
Mirosław Pęczak definiuje subkulturę jako spójną grupę społeczną, wyrażającą swoją odmienność poprzez negację utrwalonych wzorów kultury i pozostającą na marginesie dominujących tendencji życia społecznego.\footnote{M. Pęczak, str. 4} Podobną tezę stawia M. Filipak, który nacisk w swej teorii kładzie na chęć ulepszenia kultury zastanej.\footnote{M. Filipak, str. 17} Socjolog zaznacza, że celem subkulturowości nie jest obalenie dominujących wzorców kulturowych. Jednocześnie istotne jest to, by zrozumieć, że grupy subkulturowe nie powstają w próżni społecznej -- muszą istnieć w opozycji do kultury, z której się wywodzą.\footnote{T. Paleczny, Grupy subkultury młodzieżowej. Próba analizy -- propozycje teoretyczne, "Kultura i społeczeństwo" 1993, t. 37, nr 3, s. 179-190} Jeśli przyjmuje się, że potrzeba uczestnictwa w subkulturze, wywodzi się z buntu, musi istnieć zespół norm, wobec którego owa subkultura może się buntować. W ujęciu R. Dyoniziaka, pomija się jednak aspekt rewolty. W tym rozumieniu, subkultura jest niczym innym, jak zbiorem wartości, norm i wzorów, wytworzonych przez wiele jednostek mających podobne problemy i wspólne zainteresowania i dążenia, połączonych trwałą więzią.\footnote{R. Dyoniziak, Młodzieżowa podkultura, 1965, Warszawa, str. 11} 

Słowa "subkultura" używano zatem do określania zarówno grup przestępczych, etnicznych, zawodowych, religijnych, ekonomicznych, jak i ruchów społecznych. Jest to możliwe dzięki istnieniu czynnika wspólnego dla wszystkich tych społeczności -- pomimo odmiennych form uczestnictwa w kulturze, łączy je negatywny stosunek do kultury dominującej.\footnote{M. Pęczak, Mały słownik subkultur młodzieżowych, str. 4}

Niezależnie od tego, jaką metodę zdefiniowania subkultury przyjmiemy, z analizy powstałych do tej pory w dziedzinie nauk społecznych, teorii, wynika, że subkultura powinna charakteryzować się odrębnością od innych grup społecznych i negacją kultury dominującej lub jej elementów, np. obowiązujących norm; a także spójnością, w obrębie której rozumie się wspólne zainteresowania i dążenia, normy i przyjmowane wartości. 

%Część badaczy odnajduje również pozytywne aspekty uczestnictwa w subkulturach. Okazuje się, że grupy subkulturowe sprzyjają walce o własne ideały i mogą mieć wartości edukacyjne dla młodych jednostek wkraczających w dorosłe życie. 

%na koniec to raczej
 %pejoratywne znaczenie terminu. 


\section{Geneza subkultury heavymetalowej}
%zastanowić się nad przeniesieniem tego podrozdziału na sam koniec
"Metalowcy" tworzą subkulturę fanów muzyki heavymetalowej, będącej odmianą rocka.\footnote{Mirosław Pęczak, Mały słownik subkultur młodzieżowych, Warszawa 1992, Semper, str. 53}  Największe znaczenie i~popularność, zyskali w~latach 80-ych, choć początki istnienia subkultury można było zaobserwować już dziesięć lat wcześniej.\footnote{Marian Filipiak, Od subkultury do kultury alternatywnej, Lublin 2003, Wydawnictwo UMCS, str. 78} 

W Polsce, pojawili się wraz z pierwszymi ugrupowaniami muzycznymi grającymi muzykę z gatunku heavy metal (na przykład KAT i~TSA).\footnote{Witold Wrzesień, Krótka historia młodzieżowej subkulturowości, Warszawa 2013, PWN, str. 328} Już wtedy, spotykali się z krytycznym nastawieniem ze strony polskiego społeczeństwa. Oskarżano ich o tendencje satanistyczne i~neopogańskie, oraz wyrażano głęboką dezaprobatę dla wykonywanego przez nich rodzaju muzyki.\footnote{Marian Filipiak, Od subkultury do kultury alternatywnej, Lublin 2003, Wydawnictwo UMCS, str. 78} % Zachowania polskich zespołów heavymetalowych umacniały przeciwników subkultury w~tych osądach. Za przykład mogą posłużyć podpalenie i~profanacja symboli religijnych na scenie, podczas festiwalu w~Jarocinie w~1986 roku.\footnote{Witold Wrzesień, Krótka historia młodzieżowej subkulturowości, Warszawa 2013, PWN, str. 328}
Podstawę do nietolerancji od zawsze tworzył także wygląd fanów muzyki metalowej. Poza nietypowym brzmieniem i~kontrowersyjnym zachowaniem scenicznym metalowych wykonawców, również ubiór i~fryzury, formowały przestrzeń społeczną wychodzącą poza ramy koncertowych hal, i~prowokowały negatywne reakcje tych, którzy do niej nie należeli. 
 
Styl heavymetalowców to przede wszystkim suma adaptacji wybranych elementów wyglądu zewnętrznego punków i~motocyklistów.\footnote{Witold Wrzesień, Krótka historia młodzieżowej subkulturowości, Warszawa 2013, PWN, \break str. 323-324} Charakterystyczną \break część garderoby stanowią dla nich dżinsowe katany oraz skórzane kurtki motocyklowe (zwane potocznie ramoneskami od nazwy zespołu punkrockowego Ramones). W połączeniu z ciasnymi, dżinsowymi lub skórzanymi spodniami oraz długimi włosami, tworzą podstawę wyglądu heavymetalowców. Równie elementarne dla fanów metalu są pieszczochy -- skórzane bransolety, zwykle nabijane rzędami stalowych ćwieków. W zależności od ich ilości, pieszczocha pokrywa jedynie nadgarstek lub nawet całe przedramię właściciela. Popularną ozdobą ubioru są naszywki z logotypami zespołów metalowych, a~także żelazne krzyże. T-shirty, najczęściej czarne, mogą zawierać emblematy z okładek ich płyt\footnotemark[\value{footnote}] oraz rozpiskę trasy koncertowej. %pociągnąc o koncertach tutaj?
Z czasem, wśród fanów metalu powszechnym obuwiem stały się glany (ciężkie buty robocze z metalowymi noskami). 

Tak zarysowany portret metalowca przywodzi na myśl osobnika męskiego,  ale zaprezentowany wizerunek wspólny jest zarówno dla mężczyzn, jak i~kobiet przynależących do subkultury heavymetalowej. Strój kobiecy niemal niczym nie odbiega od ubioru mężczyzn, a~o samej subkulturze mówi się, jakoby zakładała kult siły i~męskości.\footnotemark[\value{footnote}] Nie wpływa to, a~przynajmniej nie w~znaczącym stopniu na ilość fanek metalu. Choć uważa się, że grupą docelową omawianego gatunku muzyki są młodzi mężczyźni, a~żeńskich wykonawców można zaliczyć do zdecydowanej mniejszości, nie istnieją statystyki wykluczające kobiety z grona odbiorców metalu. Właściwie, Barbara Major sugeruje, że współcześnie publiczność heavy metalu jest publicznością mieszaną, mężczyznom przypisując po prostu większą aktywność w~ramach działań subkulturowych.\footnote{Barbara Major, Dionizos w~glanach. Ekstatyczność muzyki metalowej, Kraków 2013, Księgarnia Akademicka, str. 48} 

% Wspomniana wyżej siła heavy metalu %tu coś innego by się przydało przejawia się nie tylko w~stroju jego fanów i~wykonawców, czy ciężkich riffach. Jest obecna także w~heavymetalowych nazwach zespołów, które manifestują ich energię. Grupy metalowe często w~swoim nazewnictwie nawiązują do podmiotów kojarzonych z siłą i~niebezpieczeństwem (Scorpions, Iron Maiden, Poison), czy nawet śmiercią (Megadeath, Slayer).\footnote{Robert Walser, Running with the Devil, str. 2} Heavy metal nierozerwalnie łączy moc z destrukcją, ale przez niektórych jego "demoniczność", "sataniczność" jest uważana jedynie za formę ekspresji scenicznej.\footnote{Barbara Major, blabla, str. 133-135}

Uczestnictwo w~koncertach jest dla heavymetalowców najważniejsze. Deena Weinstein podkreśla, że to podczas wydarzeń muzycznych, subkultura metalowa objawia się w~idealnej postaci, a~koncert heavymetalowy badaczka przyrównuje do celebracji na miarę ceremonii religijnych.\footnote{Deena Weinstein, str. 223-231} Poza tą sferą, metalowcy nie posiadają własnych rytuałów; to koncerty i~festiwale muzyczne tworzą dla nich przestrzeń, w~której najlepiej mogą manifestować swą subkulturowość.\footnote{Witold Wrzesień, Krótka historia młodzieżowej subkulturowości, Warszawa 2013, PWN, \break str. 325} Jak sugeruje Barbara Major, wspólnotowość heavy metalu ugruntowana jest w~muzyce.\footnote{Barbara Major, Dionizos w~glanach. Ekstatyczność muzyki metalowej, Kraków 2013, Księgarnia Akademicka, str. 150} Koncert tworzy okazję do poznania nowych ludzi oraz, być może przede wszystkim, wymiany doświadczeń i~nazw nowych, heavymetalowych kapel.\footnote{Magdalena Waga, Jak cię widzą, tak cię piszą... - czyli o metalowcach, www.epiotrkow.pl\break /artykul/Jak-cie-widza-tak-cie-pisza...--czyli-o-metalowcach,4963} Spajającym subkulturę czynnikiem jest ten sam gust muzyczny, a~nie prezentowane poglądy - metalowcy nie wyznają bowiem specyficznej tylko dla nich ideologii.\footnote{Marian Filipiak, Od subkultury do kultury alternatywnej, Lublin 2003, Wydawnictwo UMCS, str. 78}

Charakterystyczne dla koncertów heavymetalowych są podejmowane przez ich uczestników formy zabawy.\footnote{Witold Wrzesień, Krótka historia młodzieżowej subkulturowości, Warszawa 2013, PWN, \break str. 324} Jedną z najpopularniejszych i~jednocześnie najbardziej sztampową jest tzw. headbanging, czyli energiczne machanie, kręcenie głową w~rytm muzyki. Do tej pory, doczekała się w~społeczności heavymetalowej kilku odmian. \textit{Head banging w~kontekście muzyki metalowej, praktykowany podczas koncertu, wskazuje na przynależność do pewnej wspólnoty, jest również gestem świadczącym o aprobacie dla słuchanej właśnie muzyki, całkowitym poddaniu się jej.}\footnote{Barbara Major, str. 53} W swej prostocie, zawiera komunikat dla występujących na scenie artystów. Bynajmniej nie jest jedynie formą tanecznej ekspresji. Ze względu na silne subkulturowe znaczenie tego rodzaju tańca, a~także jego szeroką rozpoznawalność również w~kręgach \textit{niemetalowych}, od headbangingu wywodzą się anglojęzyczne nazwy subkultury -- \textit{headbangers} i~\textit{metalheads}.\footnote{Witold Wrzesień, Krótka historia młodzieżowej subkulturowości, Warszawa 2013, PWN, \break str. 325} 

Innym rodzajem tanecznej aktywności metalowców jest pogo, typowe również dla subkultury punkowej. Jego uczestnicy wzajemnie odbijają się od siebie, skacząc i~energicznie potrząsając przy tym głową.\footnote{Barbara Major, Dionizos w~glanach..., Kraków 2013, Księgarnia Akademicka, str. 158} Przestrzeń, w~której odbywa się omawiany tutaj zespół zjawisk ruchowych, zwykle zwie się \textit{kotłem} bądź \textit{młynem}; i~znajduje się ona w~środkowej i~jednocześnie najbardziej żywo reagującej części widowni. Reprezentatywną dla metalowców odmianą pogo jest tzw. \textit{mosh}, w~którym bardzo dużą rolę odgrywają ruchy głowy. W \textit{moshingu} dozwolone jest używanie pieszczoch oraz glanów. Uważa się, że ten rodzaj pogo jest bardziej agresywny niż pogo punkowe, ale jednocześnie wydaje się mniej dynamiczny.\footnote{Anna Matras, Pomiędzy ciałem indywidualnym a~zbiorowym. Praktyki cielesne punków i~ich strategie uczestnictwa w Festiwalu w~Jarocinie, Palimpsest nr 2, marzec 2012, str. 154} Popularną formą zabawy w~pogo jest dla subkulturowości metalowej również \textit{ściana śmierci}. Jej uczestnicy stają na przeciw siebie w~dwóch rzędach i~w~odpowiednim momencie wykonywanego przez grający zespół utworu, ruszają wprost na siebie.\footnote{Barbara Major, Dionizos w~glanach..., Kraków 2013, Księgarnia Akademicka, str. 158} Ponownie łączą się w~tańcu w~umownie przyjętym centrum, formułując jeden podskakujący w~rytm muzyki organizm. Z zewnątrz, praktykowana forma zabawy przypominać może rodzaj chaotycznej bójki i~przywoływać na myśl skojarzenia związane z agresją, choć ta stanowi co najwyżej formę rozładowania napięcia i~kumulującej się w~uczestnikach koncertu energii. Jeffrey Jensen Arnett w~swojej rozprawie na temat metalu konstatuje, że choć owe rytuały, dla kręgu osób spoza społeczności metalowej mogą wydawać się kuriozalne, to nie są one tak bardzo różne w~swej formie od ceremoniałów praktykowanych od wieków na całym świecie.\footnote{Jeffrey J. Arnett, Metalheads: Heavy Metal Music and Adolescent Alientation, Colorado 1996, Westview Press, str. 12} 

Przy omawianiu subkultury heavymetalowej, należy zauważyć, że dla tej społeczności, metal jest nie tylko gatunkiem muzycznym. Bycie fanem muzyki metalowej wiąże się z zaangażowaniem oraz przyjęciem określonego sposobu życia; nadaje mu znaczenie i~kreuje poczucie przynależności.\footnote{William Phillips, Brian Cogan, ..., str. 5-6} Rezygnacja z uczestnictwa w~subkulturowości wiązać się może z rezygnacją z metalu, a~to, jak przekonuje Barbara Major, traktowane jest jak świętokradztwo.\footnote{Barbara Major, Dionizos w~glanach, blabla, str. 130} W metalu nacisk kładzie się na prawdziwość -- prawdziwą muzykę i~prawdziwych heavymetalowców, co wielokrotnie podkreślane jest przez współtwórców subkultury. Niestety, nie istnieje jedna definicja dla przytaczanej tutaj prawdziwości, co generuje problemy z ostatecznym wyznaczeniem tego, co według wspólnoty heavymetalowej można uznać za \emph{metalowe}, a~czego w~te ramy wpisywać się nie powinno. Grozi to pojawianiem się niesnasek pomiędzy jej członkami, zwłaszcza, że kluczem do identyfikacji fana metalu i~odróżnienia go od pozera, od zawsze była różnica pomiędzy tym, co autentycznie metalowe, a~tym, co rzekomo za metalowe się uznaje.\footnote{Encyclopedia of Heavy Metal Music, str. 5} Wydaje się, że sformułowanie definicji owej \emph{metalowości}, mogłoby być przydatne obecnie bardziej niż kiedykolwiek, gdy subkulturowy styl  manifestowany jest głównie podczas wydarzeń muzycznych i~rzadko wychodzi poza ich ramy, a~inne aspekty tej subkulturowości przetrwały tylko w~małych grupach największych pasjonatów.\footnote{Witold Wrzesień, Krótka historia młodzieżowej subkulturowości, Warszawa 2013, PWN, str. 328}

\section{Historia heavy metalu na świecie}
Powstanie terminu "heavy metal" datuje się nawet na dwieście lat wstecz. W XIX wieku, pod pojęciem heavy metalu rozumiano ciężką artylerię wojskową, a~wyrażenie \textit{a man of heavy metal} oznaczało człowieka silnego, posiadającego władzę.\footnote{Robert Walser, Running with the Devil: Power, Gender, and Madness in Heavy Metal Music, 2014, Wesleyan University Press}  Obecnie, według \textit{Cambridge Dictionary} wyróżnia się jego dwa znaczenia: 
\begin{enumerate}
\item w~chemii i~przemyśle metalurgicznym: ciężkie, często toksyczne metale  i~półmetale
\item w~muzyce: rodzaj muzyki rockowej o mocnym brzmieniu\footnote{Cambridge Dictionary}
\end{enumerate}
%gdzieś tutaj footnote
Pochodzenie określenia "heavy metal" w~kontekście muzyki rockowej, pozostaje kwestią sporną. Istnieje kilka różnych stanowisk w~tej kwestii. Część badaczy przyjmuje, że pojawiło się za sprawą powieści Williama S. Burroughsa, w~której autor określił jednego z bohaterów mianem "the heavy metal kid". Za inne potencjalne źródło terminu podaje się utwór "Born to Be Wild" zespołu Steppenwolf.\footnote{Barbara Major, Dionizos w~glanach..., str. 41} Niezależnie od tego, jaki punkt widzenia przyjmiemy, termin został rozpowszechniony we wczesnych latach 70. i~szybko stał się nazwą dla nowego, cięższego gatunku muzyki rockowej. 

 
Muzykolog Robert Walser konstatuje, że u samego źródła metalu leżą blues i~muzyka klasyczna.\footnote{Encyclopedia of Heavy Metal Music, str. 7} %coś wspomnieć o tych dwóch gatunkach
Historię metalu łączy również z wybitnymi gitarzystami: Erikiem Claptonem i~Jimim Hendrixem, wskazując jako czynnik wiążący ich wirtuozerię w~używaniu przesterowanej gitary oraz ciężkich basów.\footnote{Robert Walser, Running with the Devil, str. 9-?} Temu drugiemu przypisuje się autorstwo pierwszego heavymetalowego hitu, "Purple Haze" z 1967 roku. Wyodrębnienie się heavy metalu jako gatunku muzycznego nie nastąpiło jednak nagle. To płynny, złożony proces, którego ramy czasowe nie są oczywiste do wyznaczenia. Początkowo, z muzyki rockowej wyosobnił się tzw. acid rock (którego nazwa odnosi się do efektów zażycia substancji halucynogennej LSD). Dopiero w~późnych latach 60. pojawiła się wczesna muzyka heavymetalowa, będąca zapowiedzią zupełnie nowego gatunku.\footnote{Jeffrey J. Arnett, Metalheads: Heavy Metal Music and Adolescent Alientation, Colorado 1996, Westview Press, str. 43} Choć zerwała z konotacjami z wyżej wymienionym narkotykiem, z acid rocka wywodziło się jej mocne brzmienie, które uczyniła mroczniejszym i~bardziej dysonansowym. \textit{Właściwie, gitarzyści heavymetalowi, jak wszyscy inni innowacyjni muzycy, kreują nowe brzmienia, czerpiąc z mocy starych, i~łącząc ich zasoby semiotyczne w~nowe, fascynujące kombinacje.}\footnote{Robert Walser, Eruptions..., str. 301} 

Za prekursorskich artystów heavymetalowych uważa się Black Sabbath oraz Led Zeppelin, \footnote{William Phillips, Brian Cogan, Encyclopedia of Heavy Metal Music, Westport 2009, ABC-CLIO, str. 28} których brzmienie oscylowało wokół prędkości i~mocy. Muzyka Led Zeppelin stanowiła sumę niespotykanych do tej pory rytmicznych wzorców i~riffów, a~sami wykonawcy stanowią dla wielu fanów niepodważalnych, fundamentalnych mistrzów heavy metalu. Z tym, i~podobnymi twierdzeniami, nie zgadzają się jednak artyści, którzy ostrożnie podchodzą do kategoryzowania swoich utworów jako metalowych.\footnote{Walser, Running, str. 6} Obawa o błędnie przypisaną  etykietę do tworzonej muzyki, była podzielana przez różnych twórców, ale wydaje się, że w~przypadku Led Zeppelin była całkowicie nieuzasadniona. Wraz z Black Sabbath, a~także niewymienionym do tej pory Deep Purple, w~gatunek heavymetalowy wpisywali się nie tylko w~warstwie muzycznej, ale także w~tekstowej i~wizualnej.\footnote{Barbara Major, Dionizos w~glanach, str. 42} W tekstach swoich utworów odwzorowywali świat pełen mistycyzmu, nie stroniąc od tematyki okultystycznej i~zjawisk nadnaturalnych. Wpłynęli również na wykształcony w~późniejszych latach styl ubioru i~sposób postrzegania świata przez fanów i~wykonawców metalu. 

Wykonawcy, którzy rodzaj tworzonej przez siebie muzyki określali jako heavy metal, pojawili się w~połowie lat 70.\footnote{Barbara Major, Dionizos w~glanach, str. 42} Początkowo, muzycy nie zyskiwali należnego im rozgłosu i~uznania. Utwory heavymetalowe nie były emitowane w~stacjach radiowych ze względu na obawę przed demoralizowaniem młodzieży.\footnote{Robert Walser, Running with the Devil, } Nie przeszkodziło im to jednak w~zgromadzeniu grup oddanych fanów o mocnym poczuciu odrębności. Do drugiej fali artystów metalowych należały takie zespoły, jak AC/DC, Kiss, Scorpions i~Judas Priest. Ten ostatni, obok Black Sabbath i~Led Zeppelin, w~latach siedemdziesiątych odniósł najwięcej lukratywnych sukcesów.\footnote{Jeffrey J. Arnett, Metalheads: Heavy Metal Music and Adolescent Alientation, Colorado 1996, Westview Press, str. 43}

Sam metal, sukces komercyjny odniósł w~latach 80. uważanych przez muzykologów za złoty okres tego rodzaju muzyki. To wtedy również, heavy metal ostatecznie wykształtował się jako dystynktywny gatunek, różny od hard rocka. Do roku 1989 płyty heavymetalowe stanowiły 40 procent płyt sprzedanych w~Stanach Zjednoczonych.\footnote{Robert Walser, Running with the Devil, str. 3} Barbara Major zauważa, że wzrost popularności sceny metalowej, ma związek z pojawieniem się glam i~pop metalu. Dzięki złagodzeniu brzmienia, utwory kwalifikujące się do jednego z tych dwóch podgatunków, trafiały w~gust szerszego grona odbiorców.\footnote{Barbara Major, str. 42} Niestety, sprzyjało to fragmentaryzacji gatunku muzycznego. 

Zbiegło się to z czasem, w~którym na rynku pojawiły się nowe technologie, dające muzykom możliwość osiągnięcia jeszcze bardziej mrocznego, szorstkiego brzmienia.\footnote{Jeffrey J. Arnett, Metalheads: Heavy Metal Music and Adolescent Alientation, Colorado 1996, Westview Press, str. 43} Dla ugrupowań tworzących muzykę agresywniejszą, bardziej ponurą niż poprzednie zespoły heavymetalowe, powstały nowe terminy opisujące ich twórczość: "speed" lub "trash" metal. Wśród nich wyróżnić można takie zespoły, jak popularne do dziś Metallica i~Megadeath. Najbardziej ekstremalnym podgatunkiem heavy metalu stał się wkrótce death metal, którego teksty utworów skupiały się niemal wyłącznie na śmierci i~przemocy. Stały one w~opozycji do twórców bardziej stonowanych odmian metalu i~skupiały węższe grono odbiorców ze względu na zniekształcone dźwięki i~chrapliwy wokal.

W latach 90. podział heavy metalu na inne gatunki muzyczne trwał nadal. Pod wpływem popularnych w~tamtym okresie hip-hopu, grunge'u i~techno, powstał nu metal,\footnote{Witold Wrzesień, str. 327} znany także jako nu tone. Swoje brzmienie zawdzięczał takim zespołom jak Korn czy Deftones, których popularność i~potencjał komercyjny pozwoliły na dalszy rozwój nurtu. Jednak ze względu na coraz większą komercjalizację, kariera zespołów nu metalowych podupadła w~latach dwutysięcznych.\footnote{Rockopedia, Nu metal, http://rockers.com.pl/rockopedia/gatunek/20.html} 

Heavy metal z mainstreamowego nurtu muzyki zniknął po odniesionym przez niego sukcesie w~latach 80. Obecnie, wpisuje się w~bardziej w~trend muzyki alternatywnej i~nie króluje już na czołowych miejscach muzycznych notowań. Nie oznacza to jednak, że zespoły metalowe nie mogą liczyć na zdobycie sławy. Wszak, Metallica, Iron Maiden czy też Judas Priest, do dziś koncertują na całym świecie, i~to nie dla kilkutysięcznej, ale zwykle nawet dla kilkudziesięciotysięcznej publiczności. 
% Wśród podstawowych instrumentów ugrupowań metalowych można wymienić: gitarę prowadzącą, gitarę basową, gitarę rytmiczną oraz perksuję (nierzadko również instrument klawiszowy).\footnote{Barbara Major, str. 48}

\section{Historia heavy metalu w~Polsce}
Heavy metal w~Polsce pojawił się na początku lat 80-ych, wraz ze wzrostem popularności muzyki rockowej w~kraju.\footnote{Mirosław Pęczak} Był to czas, w~którym na świecie muzyka z gatunku heavy metal przeżywała tak zwany "złoty okres". 

Prekursorem polskiego metalu stał się zespół TSA, założony w~Opolu za sprawą gitarzysty Andrzeja Nowaka. Do grona inicjatorów cięższego grania na polskiej scenie rockowej, zalicza się także powstałe później grupy TURBO oraz KAT, które wraz z TSA, zdaniem fanów tworzyły \textit{wielką trójcę polskiego heavy metalu}. \footnote{dorosledzieci.cba.pl/artykuly.htm} Pierwszy album studyjny ugrupowania TSA, ukazał się w roku 1983. Lata 80. przyniosły zespołowi największy sukces i w karierze muzyków stanowiły bezsprzeczny szczyt ich popularności. Nagrania TSA sprzedawały się w Polsce w kilkutysięcznych nakładach, a utwory "51", "Trzy zapałki" oraz "Zwierzenia kontestatora" uplasowały się na wysokich pozycjach licznych list przebojów.\footnote{http://www.polskirock.art.pl/tsa,z127,biografia.html [dostęp z dn. 02.04.2018]}

Początkowo, polscy artyści inspiracji szukali w dokonaniach muzycznych popularnych zespołów z zagranicy, takich jak AC/DC i Led Zeppelin. 

\chapter{Wizerunek subkultury heavymetalowej w~mediach}
\section{Czym jest wizerunek?}
Człowiek postrzega elementy rzeczywistości i przetwarza je na podstawie własnych uczuć, doświadczeń, opinii i faktów. Taki zbiór przekonań i myśli danej osoby o obiekcie (którym w tym kontekście może być zarówno firma, jednostka organizacyjna, miejsce, grupa społeczna lub osoba), określa się mianem wizerunku.\footnote{P. Kotler, Marketing: analiza, planowanie, wdrażanie i kontrola, Warszawa 1994, Gebethner \& Ska, s. 549.} Podobną definicję wizerunku proponuje również K. Huber, który charakteryzuje go jako \textit{twór wielowarstwowy, stanowiący sumę wszystkich spostrzeżeń, obserwacji, w których dokonujemy projekcji naszego ego.}\footnote{K. Huber, Image, czyli jak być gwiazdą na rynku, Businessman Book, Warszawa 1994, s. 26} 

Pojęcia wizerunku używa się zamiennie ze słowem image, wywodzącym się z łacińskiego \textit{imagio} oznaczającego obraz i symbol. Image to wyobrażenie jakiegoś ideału lub stanu, wyobrażenie danego podmiotu za pomocą zmysłów, obraz myślowy. Dziś definiowany jest również jako sposób postrzegania określonej osoby przez społeczeństwo i media.\footnote{M. Urbaniak, Wizerunek dostawcy na rynku dóbr produkcyjnych, Łódź 2003, Wydawnictwo Uniwersytetu Łódzkiego, s. 11} W ujęciu słownikowym, wizerunek oznacza podobiznę danej osoby utrwaloną np. na rysunku lub zdjęciu, a także sposób postrzegania i przedstawiania tej osoby lub rzeczy.\footnote{Słownik języka polskiego, PWN, https://sjp.pwn.pl/sjp/;2579940 [02.04.2018]}

Każda jednostka, celowo lub nie, wytwarza swój obraz, który składa się na wizerunek, definiowany w literaturze przedmiotu również jako zbiór wybranych technik komunikacyjnych, który powstaje w wyniku zamierzonych i niezamierzonych decyzji.\footnote{E. Pluta, Public relations - moda czy konieczność? Teoria i praktyka, Twigger, Warszawa 2001, s. 32} Powstały w świadomości uczestników komunikacji obraz może być odebrany pozytywnie lub negatywnie, a odbiór ten będzie się najprawdopodobniej różnił w zależności od środowiska, z jakiego pochodzą odbiorcy oraz od wyznawanych przez nich poglądów. Sam Black twierdzi, że na podstawie wizerunku -- ogólnego wrażenia odbieranego przez kogoś z zewnątrz, dokonuje się oceny, czy instytucja lub dana persona jest dobra, czy zła.\footnote{S. Black, Public relations, Warszawa 2000, Dom Wydawniczy ABC, s. 96.} Ocena ta jest oczywiście subiektywnym postrzeganiem danego obserwatora i nie musi stać w zgodzie z faktycznym stanem rzeczy. Wizerunku nie należy bowiem traktować jako wiernego odwzorowania rzeczywistości, a jako jej odbicie w świadomości określonej grupy lub jednostki. Istnieje zatem podstawa do formowania się różnic pomiędzy tym, jak dana jednostka postrzega samą siebie, a jak odbierana jest przez innych ludzi. Wizerunek jest płynny i niestały; pod wpływem zachodzących w czasie różnych czynników, może ulec znaczącym zmianom. Jak podkreśla Michał Grech, będący wizerunkiem konstrukt, jako, że wynika z procesu komunikacji, jest stale modyfikowany i negocjowany.\footnote{M. Grech, Badanie wizerunku: ludzie, marki, branże, Łódź 2012, ??, s. 12.} W związku z tym, utrzymanie stałego, zamierzonego wizerunku nie jest rzeczą łatwą. 

Postawy, zachowania, styl komunikowania się, mowa ciała i~wygląd zewnętrzny stanowią elementy, z których składa się wizerunek.\footnote{P. Czaplińska, Strategia budowania wizerunku osób znanych, [w:] Perswazyjne wykorzystanie wizerunku osób znanych, (Red.) A. Grzegorczyk, Wyższa Szkoła Promocji, Mediów i Show Businessu, Warszawa 2015, s. 30.} Przeważają wśród nich komunikaty niewerbalne, uzupełniające komunikaty słowne o całą gamę nowych informacji, których nie można byłoby pozyskać tylko z wypowiedzi DANEJ osoby. Na kreowanie wizerunku wpływają jednak także nawiązywane przez ową jednostkę kontakty międzyludzkie. Image w znaczącym stopniu zależny jest od tego, w jakim środowisku jednostka przebywa, i jakie łączą ją z nim relacje.\footnote{P. Czaplińska, op. cit., s. 30.}

%Czasy teraźniejsze uważa się za wiek wizerunku, ze względu na jego wzrastającą w ostatnich latach rolę.\footnote{P. Czaplińska, Strategia budowania wizerunku osób znanych, [w:]  Perswazyjne wykorzystanie wizerunku sób znanych, (Red.) A. Grzegorczyk, Wyższa Szkoła Promocji, Mediów i Show Businessu, Warszawa 2015} Budowanie wizerunku ma szczególne znaczenie dla celebrytów, ale 

Dla wielu subkultur, w tym subkultury metalowej, styl ubierania się jest podstawowym środkiem wyrazu własnego "ja". Część członków subkultur wyraża je poprzez, przynajmniej pozorne, nieprzywiązywanie wagi do własnego wyglądu. Sprawiają wrażenie niezadbanych, co w ich przypadku może stanowić kolejny sposób na negację i bunt. Gregory P. Stone podkreśla, że strój stanowi dla człowieka narzędzie do informowania o wartościach, jakie wyznaje, o tym, jaką przejawia postawę i nastrój -- nadaje do innych ludzi komunikat o jego tożsamości.\footnote{Gregory P. Stone, Appearance and the self: a slightly revised version, 1981, str. 193} David Muggleton dodaje jednak, że nadmierne skupianie się na swoim wizerunku jest w obrębie subkultury źle widziane i świadczyć może o zbyt nachalnej próbie zaprezentowania swojej subkulturowej przynależności. O niej najlepiej powinny świadczyć przejawiane przez jednostkę postawy i zachowania.\footnote{D. Muggleton, Wewnątrz subkultury. Ponowoczesne znaczenie stylu, Kraków 2004, Wydawnictwo Uniwersytetu Jagiellońskiego, s. 113.}
\section{Muzyka heavymetalowa i~jej fani w~mediach zagranicznych}
\section{Muzyka heavymetalowa i~jej fani w~mediach polskich}
\section{Heavy metal w~social media}
\chapter{Badanie wizerunku subkultury heavymetalowej}
\section{Przedmiot badań i~cele badawcze}
\section{Metody i~narzędzia badawcze}
\section{Analiza wyników badania}
\chapter*{Zakończenie}
\addcontentsline{toc}{chapter}{Zakończenie}
\chapter*{Bibliografia}
\addcontentsline{toc}{chapter}{Bibliografia}

\end{document}
